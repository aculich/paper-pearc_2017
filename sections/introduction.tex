\section{Introduction}

Over the last several years science has seen a large growth in the use of
computationally intensive and data-centric methods. Researchers are
increasingly using programming languages such as R or Python, and utilizing
complex algorithms in applied statistics and machine learning in order to
perform their research.

With an increased focus on computational methods
comes new challenges in teaching these techniques and sharing
knowledge with fellow scientists. There has been a growing interest in learning
outside of the traditional classroom setting. This is particularly true
for teaching computational and data analytic techniques,
as instructors must teach both conceptual and methodological information
simultaneously. One common way to teach this type of skills is a short,
time-bounded learning event, such as a bootcamp or short-course
\cite{wilson2016software}. These attempt to compress several
topics into an intensive learning session that is usually held over one or
several days.

As these new models for learning are being adopted, it offers
opportunities in developing new technology and models for pedagogy that is
focused around "hands-on" learning. Teaching a class in which students
must simultaneously learn conceptual information from the instructor, but also
perform work on their own laptops, poses a number of challenges for the
current standard approach to the bootcamp model of learning. Here we describe
recent work in utilizing cloud-based infrastructure to enhance the learning
experience of students, and streamline the ability of instructors to
prepare materials that focus on data analytic techniques. Our approach
utilizes recent advances in cloud- and cluster- based technology, such
as Kubernetes, Container technology, and the Jupyter framework for computing.

This article is a case-study covering our recent experience implementing this approach, using
advanced cyberinfrastructure to teach a day-long bootcamp in machine learning at
the University of California in San Francisco. We describe the
technical tool-chain and processes that were used in designing this course. This
unique design includes hosting all course materials online and providing
an interactive, online environment where students can run code via the cloud
without requiring them to download anything onto their machine.

Importantly, this stack only utilizes tools that are
open-source and freely available to
the community, uses computing services that are available at a low-cost
to instructors and institutions, and requires only a single course organizer
to have an understanding of the basic cloud computing technologies.
We will discuss the challenges in implementing this course setup effectively,
and discuss its merits and drawbacks.

\subsection{The bootcamp model of pedagogy}

<<<We should clarify language on bootcamp vs. workshop vs. time-bounded etc >>>

Time-bounded workshops often follow the same formula, roughly described here.

First, instructors develop materials on their own computers, sharing them with
participants (e.g., as a public Github repository). Often, instructors will use
formats, such as Jupyter notebooks, that interleave code, text describing the
data and computations, results, and illustrations
\cite{kluyver2016jupyter}. In the days preceding the event, organizers send
instructions to participants, such as how to download the materials and their
dependencies and how to configure the software dependencies to work on their
laptop computers. On the day of the workshop, instructors assume that
students have already followed these instructions successfully, or often hold
mini ``install-fest'' sessions that assist students in
getting their environments set up. The course itself emphasizes hands-on
learning, with students interacting with course material on their own
machines as the instructors teach.

These kinds of time-bounded bootcamps offer many advantages for learning over
longer and less-frequent courses. For example, they allow the students to focus
entirely on one topic for an extended period of time. This can be particularly
useful for material that demands a ``deeper dive'' and intensive, hands-on
work. It is also particularly useful for topics that attempt to jointly
cover both conceptual material and more ``hands-on'' tasks, because the
condensed time leaves more room for experimentation, discussion, and active
learning \citep{Bransford2000-lu, Papert1980-fh}. In addition, due to the
interactive setup students are more likely to work collaboratively and learn
from each other.

However, there are still many challenges associated with this class structure.
Students interact with material on their own, and their learning experience is
heavily dependent on the ability of each student to get started in the first
place. Because the instructional materials are developed on instructor
computers, differences between instructors' and students' computers (e.g., memory
available, operating system, etc.) can have hard-to-predict consequences such
as slow execution (e.g., due to sub-par student hardware) or a failure to
execute code (e.g. because of missing software dependencies on
students' computers). As these types of courses are often relatively short,
small delays result in a significant loss of instructional time.

A solution to mitigate many of these challenges is to offload the issue of
student-specific hardware onto a shared cloud computing platform. This approach
standardizes the experience of each student by allowing them access to a single
online resource for the duration of the class, without requiring any new
software or data to be downloaded on student computers. Next we describe
an implementation of this model at a day-long workshop hosted at the
University of California, San Francisco.
