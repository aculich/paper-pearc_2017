\section{Introduction}

Over the last several years, there has been a growing interest in learning
outside of the traditional classroom setting. Whether it is because departments
do not offer classes in a certain area, or because the format of semester-long
classes does not mesh well with the topics covered, there is a demand for new
approaches to teaching the skills and ideas needed to do science. One common way
to teach new skills is a short, time-bounded learning event, such as a bootcamp
or short-course \cite{wilson2016software}. These attempt to compress several
topics into an intensive learning session that is usually held over one or
several days.

These kinds of time-bounded bootcamps offer many advantages for learning over
longer and less-frequent courses. For example, they allow the students to focus
entirely on one topic for an extended period of time. This can be particularly
useful for material that demands a ``deeper dive'' and intensive working. It is
also particularly useful for topics that attempt to jointly cover both
conceptual material and more ``hands-on'' tasks, because the increased time
leaves more room for experimentation, discussion, and active learning
\citep{Bransford2000-lu, Papert1980-fh}. Finally, the opt-in nature of these
courses often ensures that students are more motivated to learn, and their
proximity both to one another as well as to the instructors makes for a good
learning environment.

Workshops often follow the same formula, roughly described here: First,
instructors develop materials on their own computers, sharing them with
participants (e.g., as a public Github repository). Often, instructors will use
formats such as Jupyter notebooks, that interleave code, text describing the
data and the computations, results, and illustrations
\cite{kluyver2016jupyter}. On the days preceding the event, instructors send
instructions to participants, including how to download the materials, and their
dependencies, and how to configure the software dependencies to work on their
laptop computers. On the day of the workshop, instructors would assume that
students have already followed these instructions successfully, or often hold
mini ``install-fest'' sessions that assist students that have problems in
getting their environments set up. These courses emphasize hands-on learning:
students interact with material on their own, and the learning experience is
heavily dependent on the ability of each student to get started in the first
place. Since instructional materials were developed on instructor computers,
differences between instructors' and students' computers (e.g., memory
available, operating system, etc.) might have hard-to-predict consequences, such
as slow execution, or even complete failure to execute.

One solution to mitigate many of these challenges is to offload the issue of
student-specific hardware onto a shared cloud computing platform. This approach
standardizes the experience of each student by allowing them access to a single
online resource for the duration of the class.

This article covers our recent experience implementing this approach, using
advanced cyberinfrastructure to teach a day-long bootcamp in machine learning at
the University of California in San Francisco. This paper describes the
technical tool-chain and process we used to host course materials online and
make them available for students to run through a browser on their own laptops,
with no additional installation required. We will discuss the challenges in
implementing this course setup effectively, and discuss its merits and
drawbacks.
