\section{Introduction}

Science has recently seen a large growth in the use of
computationally intensive and data-centric methods. Researchers are
increasingly using programming languages such as R or Python, and utilizing
complex algorithms in applied statistics and machine learning in order to
perform their research \cite{momcheva2015astro}.

With an increased focus on computational methods
comes new challenges in teaching these techniques, and new approaches
toward sharing knowledge with fellow scientists.
A rapidly growing approach to scientific training involves learning
outside of the traditional semester-long classroom setting.
This is particularly true
for teaching computational and data analytic techniques,
as instructors must teach both conceptual and methodological information
simultaneously. A common way to teach these skills is a short,
time-bounded learning event, such as a bootcamp or workshop
\cite{wilson2016software}. These  courses attempt to compress several
topics into an intensive learning session that is usually held over one or
several days.

As these new learning models are adopted, it opens
opportunities for developing new technology and models for pedagogy that are
focused on "hands-on" learning. Here we describe
recent work in utilizing cloud-based infrastructure to enhance this learning
experience, and to streamline the ability of instructors to
teach material that focuses on data analytic techniques. Our approach
utilizes recent advances in cloud- and cluster- based technology, such
as container technology (e.g., Docker) and the
Jupyter framework for computing. 

This article is a case-study covering our recent experience implementing this approach, using
advanced cyberinfrastructure to teach a multi-institutional day-long bootcamp in machine learning
hosted at the University of California in San Francisco (UCSF).
We describe the technical tool-chain and the processes that were used in
designing this course. This
unique design includes hosting all course materials online and providing
an interactive, online environment where students can run code via the cloud
without requiring them to download anything onto their machine.
It is a snapshot of the current state
of technology and practice around these ideas, and is likely to evolve rapidly
as both the tools and our knowledge of their use in pedagogy improves.

Importantly, this software stack only utilizes tools that are
open-source and freely available to
the community, and uses low-cost computing services that are available
to instructors and institutions in the United States. In addition, the
use of these technologies does not require that the entire
team of instructors become fully skilled at using them. It is
usually enough for a single team member to understand, set up, and manage these resources.
The container technology (Docker) and cloud
computing resources (XSEDE) we use are widely available platforms.
However, this teaching environment can be similarly applied using
different container or cloud computing resources.
We will discuss the challenges in implementing this course
setup effectively, and discuss its merits and drawbacks.

\subsection{The bootcamp model of pedagogy}

For the purposes of this paper we define bootcamps and workshops within
the same category of time-bounded events. These are relatively short-term
learning sessions in which a group of students moves rapidly through a
collection of training material, generally with the guidance of one
or many instructors. Time-bounded workshops often follow the same
formula, roughly described here.

First, instructors develop materials on their own computers, sharing them with
participants (e.g., as a public Github repository). Often, instructors will use
file formats, such as Jupyter notebooks, that
interleave code, text describing the
data and computations, results, and illustrations
\cite{kluyver2016jupyter}. In the days preceding the event, organizers send
instructions to participants, such as how to download the materials and their
dependencies or how to configure these dependencies on their
laptop computers. On the day of the workshop, instructors assume that
students have already followed these instructions successfully, or hold
mini ``install-fest'' sessions that assist students in
getting their environments set up. The course itself emphasizes hands-on
learning, and students interact with course material on their own
machines as the instructors teach.

This kind of time-bounded bootcamp offers many advantages for learning over
longer courses. For example, they allow the students to focus
entirely on one topic for an extended period of time. This can be particularly
useful for material that demands a ``deeper dive'' and intensive, hands-on
work. It is also particularly useful for topics that jointly
cover both conceptual material and more ``hands-on'' tasks, because the
condensed time leaves more room for experimentation, discussion, and active
learning \citep{Bransford2000-lu, Papert1980-fh}. In addition, due to the
interactive setup, students are more likely to work collaboratively and learn
from each other.

However, there are still many challenges associated with this class structure.
Students interact with material on their own, and their learning experience is
heavily dependent on the ability of each student to get started in the first
place. Because the instructional materials are developed on instructor
computers, differences between the instructor and student computers (e.g., memory
available, operating system, etc.) can have hard-to-predict consequences such
as slow execution (e.g., due to sub-par student hardware) or a failure to
execute code (e.g. because of missing software dependencies on
students' computers). These types of courses are often relatively short,
and small delays result in a significant loss of instructional time.

A solution to mitigate many of these challenges is to offload the issue of
student-specific hardware onto a shared cloud computing platform. This approach
standardizes the experience of each student by allowing them access to a single
online resource for the duration of the class, without requiring any new
software or data to be downloaded on student computers. Students can work
on course material simultaneously with the instructor and can
experiment with multiple solutions on their cloud copy of the material,
creating a more interactive teaching and learning environment. Next we describe
an implementation of this model at a day-long workshop hosted at the
University of California, San Francisco (UCSF).
