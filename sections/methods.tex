\section{Methods and technical infrastructure}\label{sec:methods}

*** THIS SECTION I'M TREATING AS METHODS-Y, IT SHOULD HAVE AS MUCH DETAIL AS POSSIBLE***

This section describes the Advanced Cyberinfrastructure (ACI) support that made
this workshop possible, and highlights reusable and shareable patterns to build
on for future work.

\subsection{Computational resources}

The Jetstream\cite{Stewart2015Jetstream} cloud platform (\url{http://jetstream-cloud.org/}) provided the
computational resources for the workshop. Jetstream's core capabilities include
the ability to create interactive Virtual Machines (VMs), access to remote
desktops through a web browser, and publishing VMs with a Digital Object
Identifier (DOI). Jetstream is
attractive to communities who have not been users of traditional HPC systems,
but who would benefit from advanced computational capabilities.

*** DETAILED DIAGRAM OF HARDWARE + SOFTWARE ***

Access to Jetstream is available to researchers at no cost through the
NSF-funded XSEDE\footnote{https://www.xsede.org/} (Extreme Science and
Engineering Discovery Environment) project\cite{Towns2014XSEDE}, which offers access to a plethora of
supercomputers and high-end visualization and data-analysis resources across the
country, to address increasingly diverse scientific and engineering challenges.

To obtain access, a qualified Principal Investigator writes a resource justification and submits an
allocation request. To help speed up the process of choosing and obtaining
access to the resource, many campuses have local XSEDE Campus Champions who can
facilitate quick access and help prepare an allocation request.

For the neuroimaging workshop, the local Campus Champion worked with Berkeley
Institute for Data Science (BIDS) and
eScience Institute data scientists to prepare an Education Allocation request. Below are
some key excerpts from the 1-page allocation request\footnote{\url{https://portal.xsede.org/documents/10308/29438/Jetstream+Education+Allocation+request+-+Sample/28517ffe-79fa-4e3f-98c9-b64f126a1e6b}},
which you can read in full from the list of example allocation requests:

\begin{itemize}
\item 50 Virtual Machines running simultaneously (40 students + 5 instructors +
test/spare/debug VMs)
\item Each VM will need to be a: Jetstream m1.medium VM (6 vCPUs, 16GB RAM, 60GB
  Storage)
\item Each VM will need an external IP address so students can connect remotely
  with a web browser to a Jupyter Notebook running on the machine
\item We are requesting 10,000SUs in total.
\end{itemize}


The technology we used to deploy the workshop in addition to the Jetstream cloud
platform includes Docker, Dockerhub, and the datascience-notebook docker-stacks\footnote{\url{https://github.com/jupyter/docker-stacks/}}
maintained by the Jupyter project.

\subsection{Development and environment control}

Each of the instructors initially used their own laptops to develop Jupyter
Notebook-based tutorials on computer vision and machine learning for
neuroscience, using state-of-the-art deep learning methods and software such as
Tensorflow and scikit-learn.

Research IT staff worked with BIDS and eScience data scientists to build a
customized container from the Jupyter project's datascience-notebook image. This
provides a pre-configured Jupyter Notebook 4.3.x; Conda Python 3.x and Python
2.7.x environments; and several common libraries including: pandas, matplotlib,
scipy, seaborn, scikit-learn, and scikit-image. Additional neuroscience-specific
packages were included such as Dipy for diffusion magnetic resonance imaging
(dMRI) analysis.

This customized container ensured that all the students had an identical
environment on the day of the workshop, including all required software
dependencies. The container made it possible for each participant to easily run
the software without installing each of the components, often a lengthy and
error-prone process at the start of many workshops. The container can also be
used as a snapshot-in-time or a ``time capsule'' so the software is preserved for
future use. Months or years for now it is possible to re-run the notebooks
again, even if external software packages and dependencies have changed.

The container image was pushed to Docker hub \footnote{\url{https://hub.docker.com/}},
which provides a centralized resource for container image discovery,
distribution and change management, user and team collaboration, and workflow
automation. At a high-level the process is:

\begin{enumerate}
  \item On a laptop, create Dockerfile that:
  \begin{itemize}
    \item derives \texttt{FROM jupyter/datascience-notebook}
    \item installs workshop-specific software packages and libraries
    \item ensure software dependencies are pinned with explicit versions defined
  \end{itemize}
  \item Build docker image
  \item Run docker image as container
  \item Test Jupyter notebooks running inside container
  \item Push docker image to DockerHub
\end{enumerate}

\subsubsection{Putting it together}


On the day of the workshop, the 50 Jetstream virtual machines (VMs) were
deployed by hand using Jetstream's Atmosphere web interface
(\url{http://www.cyverse.org/atmosphere}). While it is possible to create
scripts for a fully automated deployment using the low-level OpenStack API that
Jetstream is built on, we decided that the additional complexity was not
desirable. Using the Atmosphere web inteface is a quick and simple 6-step
process that allowed us to manually start the deployment of all 50 instances in
just a few minutes. At a high-level the process is:

\begin{enumerate}
\item Select the pre-defined VM image: Ubuntu 14.04.3 Development GUI
\item Choose instance size: m1.medium VM (6 vCPUs, 16GB RAM, 60GB
  Storage)
\item Click ``Advanced Options''
\item Select deployment script from github URL for the workshop
\item Click ``Continue to Launch''
\item Click ``Launch Instance''
\end{enumerate}

The deployment script from github URL for the workshop is a simple bash script
that will be run when the VM starts that does the following:

\begin{enumerate}
\item Install Docker on the Jetstream VM
\item Pull the workshop docker image from DockerHub
\item Download all the data needed for the workshop examples
\item start the Docker container running the Jupyter notebook, password-protected on a standard web port (80/443)
\end{enumerate}

It is worth noting a few issues related to networking and security that must be
addressed for any scenario involving remote computing (whether in a cloud
computing environment or traditional server environment).

When running Jupyter Notebooks on a laptop, they typically listen on network
port 8888 and a user connects via their web browser to
\url{http://localhost:8080}. In this configuration, the port is not accessible
to a remote attacker. Whereas in a server environment, remote access is a key
feature, so it requires running the Jupyter notebook in a secure mode requiring
a token or a password. Thankfully, the Jupyter team has configured the
docker-stacks to run in a secure mode by default.

For the workshop we chose a single password to deploy to each of the Jupyter VM
instances and wrote the password on the whiteboard for the participants. We also
copied the IP address of each machine into a Google spreadsheet, and assigned it to
each of the participants.

While it is possible to download and run a customized container directly on a
participant’s laptop, the instructors wanted to simplify the workshop
experience for the students.

A Docker container for each participant was therefore provisioned to
a virtual machine (VM) for each student and instructor running remotely on the
XSEDE Jetstream cloud platform. This allowed instructors to dive directly into
the material, without a download-and-install step. The participant only needed
to connect to their assigned IP address and type in a password provided on the
whiteboard.

After the workshop the participants were allowed to continue accessing their
notebook on the Jetstream platform for a limited time using the Education
Allocation for the workshop. After the allocation expired, each individual could
either:

\begin{itemize}

\item install Docker for Mac or Docker for Windows to download and run the
container on their own laptop

\item apply for their own Startup and Research Allocations on XSEDE Jetstream

\end{itemize}
